%!TEX root = main.tex


% Добавьте ссылку на файлы с текстом работы
% Можно использовать команды:
%   \input или \include
% Пример:
%    \input{mainfiles/1-section} или \include{mainfiles/2-section}
% Команда \input позволяет включить текст файла без дополнительной обработки
% Команда \include при включении файла добавляет до него и после него команду
% перехода на новую страницу. Кроме того, она позволяет компилировать каждый файл
% в отдельности, что ускоряет сборку проекта.
% ВАЖНО: команда \include не поддерживает включение файлов, в которых уже содержится команда \include,
% т.е. не возможен рекурсивный вызов \include
\newcommand*{\Source}{
    \include{content/1-intro}
}


% Информация о годе выполнения работы
\def\Year{%
    % 2006%
    \the\year%     % Текущий год
}

% Укажите тип работы
% Например:
%     Выпускная квалификационная работа,
%     Магистерская диссертация,
%     Курсовая работа, реферат и т.п.
\def\WorkType{%
    % Выпускная квалификационная работа%
    % Магистерская диссертация%
    % Курсовая работа%
    % Реферат%
    Отчёт о выполнении учебно-исследовательской работы студента%
}

% Название работы
%%%%%%%%%%% ВНИМАНИЕ! %%%%%%%%%%%%%%%%
% В МГУ ОНО ДОЛЖНО В ТОЧНОСТИ
% СООТВЕТСТВОВАТЬ ВЫПИСКЕ ИЗ ПРИКАЗА
% УТОЧНИТЕ НАЗВАНИЕ В УЧЕБНОЙ ЧАСТИ
\def\Title{%
    Расчёт на прочность летательного аппарата%
}


% Имя автора работы
\def\Author{%
    Алексюк Максим Александрович%
}

% Информация о научном руководителе
%% Фамилия Имя Отчество%
\def\SciAdvisor{%
    Нетелев Андрей Викторович%
}
%% В формате: И.~О.~Фамилия%
\def\SciAdvisorShort{%
    А.~В.~Нетелев%
}
%% должность научного руководителя
\def\Position{%
    % профессор%
    % доцент%
    % старший преподаватель%
    преподаватель%
    % ассистент%
    % ведущий научный сотрудник%
    % старший научный сотрудник%
    % научный сотрудник%
    % младший научный сотрудник%
}
%% учёная степень научного руководителя
\def\AcademicDegree{%
    % д.ф.-м.н.%
    % д.т.н.%
    % к.ф.-м.н.%
    % к.т.н.%
    % без степени%
}

% Информация об организации, в которой выполнена работа
%% Город
\def\Place{%
    Москва%
}
%% Университет
\def\Univer{%
    Московский авиационный институт%
}
%% Факультет
\def\Faculty{%
    Институт №6 "Аэрокосмический"%
}
%% Кафедра    
\def\Department{%
    Кафедра 601 "Космические системы и ракетостроение"%
}     

%%%% Переключите статус документа для отладки
%%%% В режиме draft документ собирается очень быстро
%%%% и выводится полезная информация о том
%%%% какие строки вылезают за границы документа, что удобно для борьбы с ними
\def\Status{%
    % draft%
    final%
}

%%%% Включает и выключает подпись <<С текстом работы ознакомлен>>
\def\EnableSign{%
    % true%
}
